%%%%% Diagnosing Thoracic Diseases Using Machine Learning & Medical Imaging

\documentclass{article}

\usepackage{microtype}
\usepackage{graphicx}
\usepackage{subfigure}
\usepackage{booktabs}
\usepackage{hyperref}

\newcommand{\theHalgorithm}{\arabic{algorithm}}
\usepackage[accepted]{icml2024}

\usepackage{amsmath}
\usepackage{amssymb}
\usepackage{mathtools}
\usepackage{amsthm}

\usepackage[capitalize,noabbrev]{cleveref}

%%%%%%%%%%%%%%%%%%%%%%%%%%%%%%%%
% THEOREMS
%%%%%%%%%%%%%%%%%%%%%%%%%%%%%%%%
\theoremstyle{plain}
\newtheorem{theorem}{Theorem}[section]
\newtheorem{proposition}[theorem]{Proposition}
\newtheorem{lemma}[theorem]{Lemma}
\newtheorem{corollary}[theorem]{Corollary}
\theoremstyle{definition}
\newtheorem{definition}[theorem]{Definition}
\newtheorem{assumption}[theorem]{Assumption}
\theoremstyle{remark}
\newtheorem{remark}[theorem]{Remark}

\usepackage[textsize=tiny]{todonotes}

\icmltitlerunning{Diagnosing Thoracic Diseases Using ML}

\begin{document}

\twocolumn[
\icmltitle{Diagnosing Thoracic Diseases Using Machine Learning and Medical Imaging}

\begin{icmlauthorlist}
\icmlauthor{Wendy Carvalho (02026116),}{}
\icmlauthor{Meriem Elkoudi (02015993),}{}
\icmlauthor{Chris Peters (01989716),}{}
\icmlauthor{Saim Siddiqui (02018510),}{}
\icmlauthor{and Amitha Thalanki (02077527)}{}
\end{icmlauthorlist}

\vskip 0.3in

]

\begin{abstract}
    TODO: Write a concise summary of the project and the conclusions of the work. It should be no longer
    than one short paragraph (e.g. 200 words).
\end{abstract}

\section{Introduction}
Thoracic diseases, including pneumonia, emphysema, and fibrosis, are common causes of morbidity and
mortality worldwide. Chest X-rays are one of the most widely accessible and commonly used diagnostic
tools for detecting these conditions. However, interpreting these images is complex, and clinical
diagnoses are often challenging, even for experienced radiologists.

Machine learning techniques are used to address this challenge by enabling automated analysis of
medical images. Bringing this leading-edge technology's immense power into the diagnostics field can
potentially increase the capacity for trained professionals to diagnose patients properly, sooner,
and in more difficult-to-identify cases.

This project aims to build a computer-aided diagnostic (CAD) tool capable of diagnosing thoracic
diseases from chest X-ray images, assisting clinicians by automating the detection of common thoracic
diseases. This has been implemented using Support Vector Machines (SVM) and Convoluted Neural Networks
(CNN). The efficacy of both are tested.


\section{Methods}
TODO: Provide a detailed description of your method and explain why the method is a good fit for
the problem.

\section{Data}
TODO: Describe the data used for experiments and report data statistics as well as interesting observa-
tions or patterns in the data.

\section{Results}
TODO: Briefly describe the evaluation approach and metrics. Report performance metrics for the method(s) through Figures or Tables.
Report insights obtained from the results. Good ways to obtain insight are ablation analysis, error
analysis, and use of synthetic data.

\section{Conclusion}
TODO: In one short paragraph concisely summarize the main points and insights of the project,
describe potential directions to extend your project, and describe limitations of your project.

\section{Contribution Chart}
TODO: Complete the following Table to clearly report the contributions that each team
member made to the final project. (Task/Subtask, Student ID, Commentary on Contribution)


%%% TO DO: ADD BIBLIOGRAPHY AND RESOURCES %%%


\end{document}
